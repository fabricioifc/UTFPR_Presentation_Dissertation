%%%% utfpr-slides.tex, 2019/12/01
%%%% Copyright (C) 2015-2019 Luiz E. M. Lima (luizeduardomlima@gmail.com)
%%
%% This work may be distributed and/or modified under the conditions of the
%% LaTeX Project Public License, either version 1.3 of this license or (at your
%% option) any later version.
%% The latest version of this license is in
%%   http://www.latex-project.org/lppl.txt
%% and version 1.3 or later is part of all distributions of LaTeX version
%% 2005/12/01 or later.
%%
%% This work has the LPPL maintenance status `maintained'.
%%
%% The Current Maintainer of this work is Luiz E. M. Lima.
%%
%% This work consists of the files utfpr-slides.sty and utfpr-slides.tex.
%%
%% A brief description of this work is in readme.txt.

%% Detecção e aviso sobre comandos obsoletos
% \RequirePackage[l2tabu, orthodox]{nag}

%% Classe de documento e opções
%%%% Modo apresentação: descomente o comando \documentclass[]{beamer}
\documentclass[%% Opções: [*] comente para remover; [>] passada para pacotes
  10pt,%% Tamanho de fonte: 10pt, 11pt, 12pt, etc.
  aspectratio = 169,%% Razão de aspecto: 169 (16:9), 43 (4:3), etc.
  compress,%% Redução de tamanho das barras de navegação [*]
  % handout,%% Versão que usa especificações de sobreposição do folheto [*]
  t,%% Alinhamento vertical dos quadros: b (fundo), c (centro) e t (topo)
  english,%% Idioma secundário (penúltimo) [>]
  brazilian,%% Idioma primário (último) [>]
  tikz,
]{beamer}
%%%% Modo artigo: descomente o comando \documentclass[]{article}
% \documentclass[%% Opções: [>] passada para pacotes
  % a4paper,%% Tamanho de papel: a4paper, letterpaper, etc.
  % english,%% Idioma secundário (penúltimo) [>]
  % brazilian,%% Idioma primário (último) [>]
% ]{article}
% \documentclass[tikz]{article}

%% Pacotes utilizados
\usepackage[%% Opções:
  Times   = false,%% Fontes Times (roman) e Arial (sans serif): true ou false
  BibURLs = false,%% Links de URLs nas referências: true ou false
  GovLogo = false,%% Logomarca do Governo Federal: true ou false
  MECLogo = false,%% Logomarca do Ministério da Educação: true ou false
  LogosOn = false,%% Logomarcas nos títulos dos slides: true ou false
  ABNTNum = none,%% Estilo numérico ABNT: none (AUTOR, ANO), dflt (1) e brkt [1]
]{utfpr-slides}
\usepackage{color,soul}

%% Arquivo de referências
\addbibresource{utfpr-slides.bib}

%% Arquivos de logomarcas (diretório ``Logos''): comente para remover
\logoevent{evento}%% Logomarca do evento
\logoorg{evento-org}%% Logomarca da organização promotora
\logoextinst{inst-ext}%% Logomarca da instituição do autor externo
\logoprog{ppgcc-pg}%% Logomarca do programa ou do curso
% \logodept{depto}%% Logomarca do departamento ou da coordenação
\logoextra{utfpr-110anos}%% Logomarca extra (à esquerda da logomarca do câmpus)
\logocampus{utfpr-pg}%% Logomarca do câmpus

%% Informações do documento
%%%% Título: [CURTO]; {LONGO}
\title[REDES NEURAIS CONVOLUCIONAIS NA SEGMENTAÇÃO SEMÂNTICA DE IMAGENS AÉREAS PARA O MAPEAMENTO DA COBERTURA DA TERRA EM ÁREAS DE PROTEÇÃO AMBIENTAL]{%
  \bfseries%
  Redes Neurais Convolucionais na Segmentação Semântica
  \\de Imagens Aéreas para o Mapeamento da Cobertura da Terra
  \\em Áreas de Proteção Ambiental
  % Título da Apresentação em Congresso, Seminário ou Evento%
  % \\Técnico/Científico, ou para Defesa de Trabalho Acadêmico%
}
%%%% Subtítulo
% \subtitle{%
%   Subtítulo da Apresentação em Congresso, Seminário ou Evento%
%   \\Técnico/Científico, ou para Defesa de Trabalho Acadêmico%
% }
%%%% Assunto
\subject{Nome e/ou Sigla do Congresso, Seminário ou Evento Técnico/Científico}
%%%% Congresso, seminário ou evento técnico/científico:
%%%% descomente os comandos \author[]{}, institute[]{} e \titlegraphic{}
%%%%%% Autor(es): [CURTO]; {LONGO}
\author[P. M. Autor(a) et al.]{%
  Fabricio Bizotto\inst{1}%
  \athanks[0000-0000-0000-0001]{fabriciobizotto@alunos.utfpr.edu.br}{%
    Departamento, Coordenação, Programa ou Curso%
  }%
  \and Mauren L Andrade\inst{2}%
  \athanks[0000-0000-0000-0002]{autor2@dominio}{%
    Departamento, Coordenação, Programa ou Curso%
  }%
  \and Gilson A Giraldi\inst{3}%
  \athanks[0000-0000-0000-0003]{autor3@dominio}{%
    Departamento, Coordenação, Programa ou Curso%
  }%
  }
%%%%%% Instituição(ões) e e-mail(s): [CURTO]; {LONGO}
\institute[UTFPR/INST-EXT]{%
  \affil[1,2]{\utfprname, Ponta Grossa, Paraná, Brasil}%
  \and\affil[3]{Departmento de Engenharia Elétrica, Pontifícia Universidade Católica do Rio de Janeiro, RJ, Brazil}%
  \and\email[1]{fabriciobizotto@alunos.utfpr.edu.br}%
  \sep\email[2]{autor2@dominio}%
  \sep\email[3]{autor3@dominio}%
  \sep\email[4]{autor4@dominio}%
  \sep\email[5]{autor5@dominio}%
}
%%%%%% Logomarcas do evento
\titlegraphic{\eventlogos}
%%%% Defesa de trabalho acadêmico:
%%%% descomente os comandos \author[]{}, institute[]{} e \titlegraphic{}
%%%%% Autor(a): [CURTO]; {LONGO}
\author[Fabricio]{%
  Fabricio Bizotto%
  \athanks[0000-0000-0000-0000]{fabriciobizotto@alunos.utfpr.edu.br}{%
    Programa de Pós-Graduação em Ciência da Computação%
  }%
}
%%%%% Instituição: [CURTO]; {LONGO}
\institute[UTFPR-PG/DAINF/PPGCC]{%
  \affil{%
    \utfprname\ (UTFPR)%
    \and Câmpus Ponta Grossa (PG)%
    \and Departamento Acadêmico de Informática (DAINF)%
    \and Programa de Pós-Graduação em Ciência da Computação (PPG-CC)%
  }%
  \and\email{mlsguario@utfpr.edu.br}%
  \and Orientadora: Profa. Dra. Mauren L Andrade%
}
%%%%% Logomarcas da instituição
\titlegraphic{\institutelogos}
%%%% Câmpus: {SIGLA}; {NOME}
\campus{PG}{Câmpus Ponta Grossa}
%%%% Departamento, coordenação, programa ou curso: [LOGO]; {SIGLA}; {NOME}
\departament[depto]{DEPTO}{Departamento, Coordenação, Programa ou Curso}
%%%% Data: [CURTO]; {LONGO}
% \date[\today]
\date[21 de novembro de 2023]

%% Início do documento
\begin{document}

\mode<presentation>{%% Modo apresentação: página de título e sumário
  \frame{\titlepage}%
  \frame[allowframebreaks]{%
    \frametitle{\contentsname}%
    \framesubtitle{~}%
    \tableofcontents%
  }%
}

\mode<article>{\maketitle}%% Modo artigo: página de título

\section{Introdução}\label{sec:intro}

\begin{frame}{Introdução}
    \begin{itemize}
        \item Área de Proteção Ambiental (APA)
        \begin{itemize}
            \item Ferramenta de Conservação da Natureza.
            \item Destinada à proteção ambiental e ao uso sustentável dos recursos naturais.
            \item Objetivo: Conservação x Desenvolvimento Econômico.
            \item Gestão governamental, como o Instituto Chico Mendes de Conservação da Biodiversidade (ICMBio).
            \item Regulamenta as atividades humanas de acordo com as características ambientais da região.
        \end{itemize}
    \end{itemize}
    \begin{block}{Principais desafios no monitoramento da APA}
        \begin{itemize}
            \item Grandes equipes de trabalho especializada
            \item Deslocamento à regiões de difícil acesso
            \item Alto custo para manutenção das equipes
            \item Perigos associados as características de fauna e flora de cada região
        \end{itemize}
    \end{block}
\end{frame}

\begin{frame}
\begin{columns}[T]

\column{0.5\textwidth}
A Fig.~\ref{fig:apa} representa a área da APA-Petrópolis, região serrana do Rio de Janeiro.
\begin{figure}[!htb]
\centering%
\caption{APA-Petrópolis.}%
\label{fig:apa}
\mode<presentation>{\includegraphics[width = 0.8\columnwidth]{./Figuras/apa}}%% Modo apresentação: tamanho da figura
\mode<article>{\includegraphics[width = 0.4\columnwidth]{./Figuras/apa}}%% Modo artigo: tamanho da figura
\source{\textcite{Google Earth, 2023}.}
\end{figure}
\column{0.5\textwidth}
\mode<article>{\par}%% Modo artigo: quebra de parágrafo
\begin{block}{APA-Petrópolis}
\begin{itemize}
\item $\approx$ 60.000 hectares
\item Ações de conservação realizadas pelo ICMBio:
\begin{itemize}
    \item Monitoramento Ambiental
    \item Manutenção de trilhas ecológicas
    \item Recuperação de áreas degradadas
    \item Combate a incêndios
    \item Proteção de espécies ameaçadas de extinção
    \item Invasão de terras
\end{itemize}
\end{itemize}
\end{block}
\end{columns}
\end{frame}

\begin{frame}{Introdução}
\begin{itemize}
    \item A cobertura e utilização do solo \textbf{(Land Use and Land Cover - LULC)}
    % \item O desenvolvimento socioeconômico dos seres humanos tem sido fortemente apoiado pelo uso da terra, sendo a urbanização um dos principais exemplos de mudanças de ocupação do solo em todo o mundo \cite{LIU, 2018}.
    \item Compreender a relação entre cobertura e uso da terra é essencial para gerir os recursos naturais, mitigar as alterações climáticas e proteger a biodiversidade \cite{ZIN; LIN, 2018}.
    \item Mudanças antropogênicas abrangem o desflorestamento, desocupação, urbanização, alterações nos tipos de cultivo e adaptações nas práticas utilizadas em cada uso do solo, tais como, técnicas de plantio e sistemas de rotação de cultura florestal \cite{Peterson et al. (2014)}.
    % \begin{block}{Principais desafios no monitoramento da APA}
    %     \begin{itemize}
    %         \item Grandes equipes de trabalho especializada
    %         \item Deslocamento à regiões de difícil acesso
    %         \item Alto custo para manutenção das equipes
    %         \item Perigos associados as características de fauna e flora de cada região
    %     \end{itemize}
    % \end{block}
\end{itemize}
\end{frame}

\begin{frame}{Introdução}

\textbf{Sensoriamento Remoto}

    \begin{itemize}
        \item Alternativa de baixo custo
        \item Acesso a base de dados de imagens para diferentes regiões
        \item Acesso a áreas de difícil acesso via solo
        \item Estudos começam a exploraram a aplicação das Redes Neurais Convolucionais (RNC) na análise da cobertura da terra com resultados promissores \cite{HU et al., 2013), (LI et al., 2020}.
    \end{itemize}

    \begin{block}{Possibilidades}
        \begin{itemize}
            \item Imagens de satélite disponíveis gratuitamente (baixa resolução)
            \item VANT (Veículo Aéreo não Tripulado) (custo alto)
            \item Imagens Aéreas RGB da plataforma Google Earth
        \end{itemize}
    \end{block}

\end{frame}



\subsection{\textbf{Objetivo Geral}}\label{ssec:intro1}

\begin{frame}
O objetivo geral deste trabalho é o desenvolvimento de uma nova metodologia para segmentação semântica de imagens de sensoriamento remoto do Google a fim de gerar o mapa de cobertura e uso do solo da região da APA-Petrópolis, Rio de Janeiro por meio de RNC
% Esta apresentação de slides foi desenvolvida com base na classe \href{http://www.ctan.org/pkg/beamer/}{\LaTeX/Beamer~\linkicon}.
% \begin{block}{Citações e referências}
% \begin{itemize}
% \item Exemplos de referências podem ser observados nas citações:
% \begin{itemize}
% \item Implícita: \ldots\ \cite{Nriagu1988,Lamport1994,VanEkenstein1997}.
% \item Explícita: Segundo \textcite{Wizentier1992,Faina2000},\ldots
% \end{itemize}
% \item Citações e referências podem ser inseridas neste documento usando os comandos do pacote \LaTeX\ \enquote{\href{http://ctan.org/pkg/biblatex/}{biblatex~\linkicon}}.
% \item Os dados de cada referência podem ser obtidos de um arquivo \enquote{bibtex} (*.bib), geralmente na própria página de \textit{download} da referência (artigos, livros, etc.), ou no Google Acadêmico, etc.
% \item Para gerar ou editar entradas de arquivos \enquote{bibtex} (*.bib), pode-se utilizar a ferramenta \enquote{\href{http://truben.no/latex/bibtex/}{Bibtex Editor~\linkicon}} ou \enquote{\href{http://zbib.org/}{ZoteroBib~\linkicon}}, entre outras.
% \end{itemize}
% \end{block}
\end{frame}

\subsection{Objetivos Específicos}\label{sec:intro2}

\begin{frame}
Para atender o objetivo geral deste trabalho, foram propostos os seguintes objetivos específicos:

\begin{block}<+->{Objetivos Específicos}
\begin{itemize}
\item Capturar e rotular imagens de sensoriamento remoto da área de proteção ambiental de
Petrópolis, Rio de Janeiro.
\item Comparar o desempenho de redes neurais do tipo SegNet e U-NET no conjunto de dados criado.
\item Comparar diferentes funções de custo a fim de avaliar a que melhor se adapte ao conjunto
de dados.
\item Utilizar métricas para analisar e comparar o desempenho dos modelos.
\item Comparar os resultados obtidos com trabalhos relacionados.
\end{itemize}
\end{block}
\end{frame}

\subsection{Trabalhos Relacionados}\label{sec:intro3}

\begin{frame}
    1. Construção de mapas digitais a partir de imagens de satélite, utilizando a arquitetura U-Net com EfficientNet-B0 como codificador e decodificador \textbf{\cite{Khanh2021}}.
    \begin{figure}[!htb]
        \centering%
        \caption{Resultado da segmentação semântica no conjunto de dados do Google com a arquitetura proposta.}%
        \label{fig:khan01}
        \mode<presentation>{\includegraphics[width = 0.8\columnwidth]{./Figuras/khan-01}}%% Modo apresentação: tamanho da figura
        \mode<article>{\includegraphics[width = 0.4\columnwidth]{./Figuras/khan-01}}%% Modo artigo: tamanho da figura
        \source{Adaptado de \cite{Khanh2021}.}
    \end{figure}
    
\end{frame}

\begin{frame}
    2. Aplicação de RNC, na tarefa de segmentação semântica de imagens obtidas por sensoriamento remoto. Duas arquiteturas de RNC, SegNet e U-net, são aprimoradas por meio da introdução de \textit{index pooling} para melhorar essas arquiteturas, permitindo a preservação de informações espaciais cruciais durante a ampliação da resolução. \textbf{\cite{Alam2021}}.
    \begin{figure}[!htb]
        \centering%
        \caption{Resultado da segmentação semântica.}%
        \label{fig:alam2021}
        \mode<presentation>{\includegraphics[width = 0.6\columnwidth]{./Figuras/Alam2021}}%% Modo apresentação: tamanho da figura
        \mode<article>{\includegraphics[width = 0.4\columnwidth]{./Figuras/Alam2021}}%% Modo artigo: tamanho da figura
        \source{Adaptado de \cite{Alam2021}.}
    \end{figure}
    
\end{frame}

\begin{frame}
    3. Redes U-Net para segmentar estufas agrícolas de plástico em imagens de sensoriamento remoto de alta resolução. A abordagem foi dividida em três etapas: coleta e anotação de imagens, treinamento da rede U-Net e pós-processamento para remover elementos confundíveis com as estufas \textbf{\cite{Chen2021}}.
    \begin{figure}[!htb]
        \centering%
        \caption{Dificuldade em extrair estufas densamente distribuidas.}%
        \label{fig:alam2021}
        \mode<presentation>{\includegraphics[width = 0.6\columnwidth]{./Figuras/apg}}%% Modo apresentação: tamanho da figura
        \mode<article>{\includegraphics[width = 0.4\columnwidth]{./Figuras/apg}}%% Modo artigo: tamanho da figura
        \source{\cite{Chen2021}.}
    \end{figure}
    
\end{frame}

% \section{Revisão da Literatura}\label{sec:revlit}

% \subsection{Processamento Digital de Imagem}\label{ssec:revlit1}

% \begin{frame}
% % \begin{block}<+->{Formação da Imagem Digital}
% \begin{itemize}
% \item Amostragem: Resolução espacial.
% \item Quantização: Intensidade de brilho.
% \end{itemize}
% % \end{block}

% \begin{figure}[!htb]
%     \centering%
%     \caption{Imagem projetada em uma matriz bidimensional.}%
%     \label{fig:alam2021}
%     \mode<presentation>{\includegraphics[width = 0.5\columnwidth]{./Figuras/amo_qua}}%% Modo apresentação: tamanho da figura
%     \mode<article>{\includegraphics[width = 0.4\columnwidth]{./Figuras/amo_qua}}%% Modo artigo: tamanho da figura
%     \source{\cite{Gonzalez2009}.}
% \end{figure}
% \end{frame}

% \subsection{Equações, com e sem Numeração}\label{ssec:revlit2}

% \begin{frame}[fragile = singleslide]
% Uma equação como $y = a x^2 + b x + c$ pode ser inserida ao longo do texto de um parágrafo usando o ambiente \LaTeX\ \enquote{math} (\verb|$...$|).
% Por outro lado, a seguinte equação é um exemplo de equação não numerada inserida numa linha em separado usando o ambiente \LaTeX\ \enquote{displaymath} (\verb|\[...\]|).
% \begin{block}{}
% \[
% \frac{\mathrm{d} y}{\mathrm{d} x} = \gamma \operatorname{sen} x
% \]
% \end{block}
% A Eq.~\eqref{eq:fx} é um exemplo de equação inserida usando o ambiente \LaTeX\ \enquote{equation} e numerada automaticamente.
% \begin{block}{}
% \begin{equation}\label{eq:fx}
% f(x) = \frac{1}{\alpha} \int_0^L \left(\frac{x^2}{2} - \frac{x^3}{3}\right) \mathrm{d} x
% \end{equation}
% \end{block}
% Para gerar ou editar equações em \LaTeX, pode-se utilizar a ferramenta \enquote{\href{http://formulasheet.com/}{Formula Sheet~\linkicon}}, entre outras.
% \end{frame}

\section{Material e Métodos}\label{sec:matmet}

\subsection{Metodologia Proposta}\label{ssec:matmet1}

\begin{frame}
A Fig.~\ref{fig:metodologia} apresenta a metodologia utilizada para a realização dos objetivos propostos.
\begin{figure}[!htb]
\centering%
\caption{Metodologia Proposta.}%
\label{fig:metodologia}
\mode<presentation>{\includegraphics[width = 0.8\columnwidth]{./Figuras/metodologia}}%% Modo apresentação: tamanho da figura
\mode<article>{\includegraphics[width = 0.4\columnwidth]{./Figuras/metodologia}}%% Modo artigo: tamanho da figura
\source{Autoria Própria.}
\end{figure}
\end{frame}

\subsection{Estudos sobre a APA-Petrópolis}\label{ssec:matmet2}

\begin{frame}
\begin{columns}[T]
\column{0.5\textwidth}
\begin{itemize}
    \item $\approx$ 60.000 hectares (5,69\%) da Mata Atlântica.
    \item Região urbanizada.
    \item Plano de manejo que define ações e restrições.
    \item Petrópolis e municípios adjacentes.
\end{itemize}
\column{0.5\textwidth}
\begin{figure}[!htb]
\centering%
\caption{APA-Petrópolis.}%
% \label{fig:apa}
\mode<presentation>{\includegraphics[width = 0.8\columnwidth]{./Figuras/apa}}%% Modo apresentação: tamanho da figura
\mode<article>{\includegraphics[width = 0.4\columnwidth]{./Figuras/apa}}%% Modo artigo: tamanho da figura
\source{Adaptado de \textcite{Google Earth, 2023}.}
\end{figure}
\end{columns}
\end{frame}

\subsection{Configuração do ambiente de Trabalho}\label{ssec:matmet3}

\begin{frame}
\begin{columns}[T]
\column{0.5\textwidth}
\begin{itemize}
    \item \textbf{Processador}: Intel(R) Xeon(R) CPU E5-2666 v3 @ 2.90GHz 2.90 GHz;
    \item \textbf{Memória RAM}: 32 GB;
    \item \textbf{Placa de Vídeo}: NVIDIA GIGABYTE RTX 3060 EAGLE OC – 12GB dedicada;
    \item \textbf{Sistema Operacional}: Microsoft Windows 10 PRO 64 bits.
\end{itemize}
\column{0.5\textwidth}

A Tab.~\ref{tab:tools} apresenta o conjunto de ferramentas utilizado durante o trabalho.
\begin{table}[!htb]
\centering%
\mode<presentation>{\scriptsize}%% Modo apresentação: tamanho de fonte
\mode<article>{\small}%% Modo artigo: tamanho de fonte
\caption{Conjunto de Ferramentas Utilizadas.}%
\label{tab:tools}
\begin{tabular*}{\columnwidth}{@{\extracolsep{\fill}}ll}
\toprule
Python                                  & \cite{PythonManual}           \\
PyTorch                                 & \cite{pytorch}                \\
Cuda Toolkit                            & \cite{NvidiaCuda}                \\
Miniconda                               & \cite{Miniconda}                \\
ArcGis Pro (parceria com UFMS           & \cite{Arcgis}                \\
Computer Vision Annotation Tool         & \cite{cvat}                \\
\bottomrule
\addlinespace
\end{tabular*}
\source{Autoria própria.}
\end{table}

\end{columns}
\end{frame}

\subsection{Aquisição do Conjunto de Dados}\label{ssec:matmet4}

\begin{frame}
\begin{itemize}
    \item Plataforma Google Earth (ArcGis Pro).
    \item Entre Maio/2022 e Dezembro/2022.
    \item Seleção aleatória da área.
\end{itemize}
\begin{figure}[!htb]
\centering%
\caption{Demonstração da aquisição do conjunto de dados.}%
\label{fig:obter_imagens}
\mode<presentation>{\includegraphics[width = 0.6\columnwidth]{./Figuras/obter_imagens-min}}%% Modo apresentação: tamanho da figura
\mode<article>{\includegraphics[width = 0.4\columnwidth]{./Figuras/obter_imagens-min}}%% Modo artigo: tamanho da figura
\source{Adaptado de \textcite{Google Earth, 2023}.}
\end{figure}
\end{frame}

\begin{frame}
\begin{columns}[T]
\column{0.5\textwidth}
% A Tabela.~\ref{tab:separacao} apresenta a quantidade de imagens obtidas.
\begin{table}[!htb]
\centering%
\mode<presentation>{\scriptsize}%% Modo apresentação: tamanho de fonte
\mode<article>{\small}%% Modo artigo: tamanho de fonte
\caption{Separação das imagens no conjunto de dados.}%
\label{tab:separacao}
\begin{tabular*}{\columnwidth}{@{\extracolsep{\fill}}lr}
\toprule
Grupo   & Quantidade     \\
\midrule
Treinamento             & 214 ($\approx 67\%$) \\
Teste                   & 42 ($\approx 13\%$) \\
Descontinuadas          & 66 ($\approx 20\%$) \\
Total                   & 322 \\
\bottomrule
\addlinespace
\end{tabular*}
\source{Autoria própria.}
\end{table}
\begin{block}{Mais Detalhes}
    \begin{itemize}
        \item RGB (TIFF).
        \item Resolução padrão de 2048x2048 pixels.
    \end{itemize}
\end{block}
\column{0.5\textwidth}
\begin{figure}[!htb]
\centering%
\caption{Amostras do Conjunto de Dados.}%
\label{fig:obter_imagens}
\mode<presentation>{\includegraphics[width = 0.8\columnwidth]{./Figuras/amostras-min}}%% Modo apresentação: tamanho da figura
\mode<article>{\includegraphics[width = 0.6\columnwidth]{./Figuras/amostras-min}}%% Modo artigo: tamanho da figura
\source{Adaptado de \textcite{Google Earth, 2023}.}
\end{figure}
\end{columns}
\end{frame}

\subsection{Seleção e Rotulagem}\label{ssec:matmet5}

\begin{frame}
\color{black}{\textbf{Seleção}} - \color{gray}{Rotulagem}
\begin{columns}[T]
\column{0.4\textwidth}
\begin{itemize}
    \item Delimitar as regiões de interesse.
    \item Definido em conjunto com equipe do ICMBio.
    \item 8 classes
    \begin{itemize}
        \item Área Desenvolvida
        \item Floresta
        \item Sombra
        \item Área em Regeneração
        \item Solo Exposto
        \item Água
        \item Rocha
        \item Agricultura
        \item \color{red}{Piscina}
    \end{itemize}
\end{itemize}
\column{0.6\textwidth}
\begin{figure}[!htb]
\centering%
\caption{Amostras do Conjunto de Dados por Classe.}%
\label{fig:obter_imagens}
\mode<presentation>{\includegraphics[width = 1.0\columnwidth]{./Figuras/amostras_classe-min}}%% Modo apresentação: tamanho da figura
\mode<article>{\includegraphics[width = 0.8\columnwidth]{./Figuras/amostras_classe-min}}%% Modo artigo: tamanho da figura
\source{Adaptado de \textcite{Google Earth, 2023}.}
\end{figure}
\end{columns}
\end{frame}

\begin{frame}
\color{gray}{Seleção} - \color{black}{\textbf{Rotulagem}}

\begin{itemize}
    \item Auxílio de um profissional do ICMBio.
    \item Critério rígido.
    \item Ferramentas
    \begin{itemize}
        \item ArcGis (semi-automático, manual)
        \item CVAT (manual)
    \end{itemize}
\end{itemize}

\begin{figure}[!htb]
\centering%
\caption{Amostra da imagem original e da imagem rotulada.}%
\label{fig:labeled}
\mode<presentation>{\includegraphics[width = 0.8\columnwidth]{./Figuras/labeled-min}}%% Modo apresentação: tamanho da figura
\mode<article>{\includegraphics[width = 0.5\columnwidth]{./Figuras/labeled-min}}%% Modo artigo: tamanho da figura
\source{Autoria própria.}
\end{figure}

\end{frame}

\begin{frame}{Metodologia Aplicada no Desbalanceamento de Classes}

\begin{columns}[T]
    
\column{0.6\linewidth}
\begin{itemize}
    \item Método proposto por \cite{Marcatto2022}.
    \item Calcular o peso para cada classe (treinamento).
\end{itemize}

\begin{equation*}\label{eq:met:desbal}
\vcenter{\hbox{\mode<presentation>{\includegraphics[width=2cm,height=2cm]{./Figuras/239-min}}}}
\vcenter{\hbox{\mode<article>{\includegraphics[width=1cm,height=1cm]{./Figuras/239-min}}}}
\qquad\qquad
\begin{aligned}
\varphi(c) = \frac{m}{C*n^{c}}
\end{aligned}
\end{equation*}

\column{0.4\linewidth}
\begin{table}[!htb]
\centering%
\mode<presentation>{\scriptsize}%% Modo apresentação: tamanho de fonte
\mode<article>{\small}%% Modo artigo: tamanho de fonte
\caption{Pesos calculados para o conjunto de treinamento.}%% Legenda
\label{tab:tools}
\begin{tabular*}{\columnwidth}{@{\extracolsep{\fill}}lr}
\toprule
\textbf{Classe}         & \textbf{Peso}             \\ \midrule
Área Desenvolvida       &       1,1472              \\
Floresta                &       0,3832              \\
Sombra                  &       2,0468              \\
Área em Regeneração     &       0,4436              \\
Agricultura             &       1,4859              \\
Rocha                   &       2,4287              \\
Solo Exposto            &       \hl{5,2043}              \\ 
Água                    &       2,0031              \\
\bottomrule
\addlinespace
\end{tabular*}
\source{Autoria própria.}
\end{table}

\end{columns}

\noindent
em que $m$ é o número de pixels de todas as imagens de treinamento, $C$ é o número de classes, e $n^c$ é o número de pixels que pertencem à classe $c$.

\end{frame}

\subsection{Implementação e Adaptações na Rede SegNet e U-NET}\label{ssec:matmet6}

\begin{frame}

\begin{itemize}
    \item \textbf{MPSegnet} proposta por \textcite{SOUZA BRITO et al., 2021}
    \item Uma nova estratégia de multi-pooling, substituindo o max-pooling por \textit{Discrete Wavelet Transform} (DWT) e unpooling por \textit{Inverse Discrete Wavelet Transform} (IWT).
    \item Conjunto de dados utilizado nos experimentos
    \begin{itemize}
        \item 2D Semantic Labeling Contest - Potsdam 
        \item IRRG: 3 canais (IR-R-G)
    \end{itemize}
\end{itemize}

\begin{figure}[!htb]
\centering%
\caption{Arquitetura da MPSegnet.}%
\label{fig:graficoxy1}
\mode<presentation>{\includegraphics[width = 0.5\columnwidth]{./Figuras/mpsegnet-min}}%% Modo apresentação: tamanho da figura
\mode<article>{\includegraphics[width = 0.5\columnwidth]{./Figuras/mpsegnet-min}}%% Modo artigo: tamanho da figura
\source{\cite{Andre2021}}
\end{figure}

\end{frame}

\begin{frame}

\begin{itemize}
    \item \textbf{U-NET} sugerida no artigo de \textcite{Nguyen-Khanh et al. (2021)}.
    \item Combinação de \textbf{EfficientNet-B0} \cite{EfficientNet} (ENCODER) para extração de características com \textbf{U-NET} \textcite{Ronneberger et al. (2015)} (DECODER) para reconstrução do mapa de características.
    \item Conjunto de dados utilizado nos experimentos
    \begin{itemize}
        \item Imagens Aéreas (Google Earth - GE)
        \item RGB: 3 canais (\textit{Red-Green-Blue})
    \end{itemize}
\end{itemize}

\begin{columns}[T]
\column{0.5\linewidth}

\begin{table}[!htb]
\centering%
\mode<presentation>{\scriptsize}%% Modo apresentação: tamanho de fonte
\mode<article>{\small}%% Modo artigo: tamanho de fonte
\caption{Comparativo entre \textit{encoder} e funções de custo.}%
\label{tab:met:unet}
\begin{tabular*}{\columnwidth}{@{\extracolsep{\fill}}lrrrr}
\toprule
\textit{ENCODER}    & Parâmetros    & Categorical Cross & Dice & Average    \\ 
                    &               & Entropy Loss      & Loss & Loss       \\
\midrule
VGG11       & 32M         & 1.194               & 0.770             & 1.066             \\
ResNet18    & 18M         & 1.191               & 0.770             & 1.065             \\
EffNet-B0   & \textbf{4M} & \textbf{1.110}      & \textbf{0.731}    & \textbf{0.997}    \\
EffNet-B1   & 6.5M        & 1.374               & 0.806             & 1.204             \\
EffNet-B2   & 8M          & 1.134               & 0.747             & 1.018             \\
\bottomrule
\addlinespace
\end{tabular*}
\source{Adaptado de \textcite{Nguyen-Khanh et al. (2021)}.}
\end{table}

\end{columns}
\end{frame}

\subsection{Treinamento e Teste}\label{ssec:matmet7}

\begin{frame}
\begin{itemize}
    \item Conjunto de dados dividido em treinamento e teste (Ver Tabela \ref{tab:separacao}).
    \item Pré-processamento: Protocolo de \textbf{janelas deslizantes} adotado também no trabalho de \cite{Andre2021}.
    \item \textbf{10.000} amostras de \textbf{256x256}.
    \item Inicialização dos pesos: Pesos pré-treinados \cite{Imagenet2009}.
\end{itemize}

\begin{columns}[T]
\column{0.4\linewidth}

\begin{block}{Treinamento}
    \begin{itemize}
        \item Conjunto de dados embaralhado.
        \item Aumento de dados
        \begin{itemize}
            \item \textbf{Espelhamento horizontal e vertical}
        \end{itemize}
    \end{itemize}
\end{block}

\column{0.5\linewidth}
\begin{block}{Teste}
    \begin{itemize}
        \item Conjunto de dados \textbf{não} é embaralhado.
        \item Sem aumento de dados.
        \item Protocolo de janelas deslizantes com passo de 32 pixels para evitar inconsistências na segmentação, especialmente nas bordas \cite{Farhangfar2019}.
        \item Maior esforço computacional.
    \end{itemize}
\end{block}

\end{columns}
\end{frame}

\begin{frame}
% \frametitle<presentation>{Frame Title Should Be in Uppercase.}
\framesubtitle{Treinamento e Teste - Mais detalhes de implementação.}

\begin{itemize}
    \item \textbf{Número de classes}: 8
    \item \textbf{Épocas de treinamento}: 100
    \item \textbf{Tamanho do \textit{batch}}: 8
    \item \textbf{Taxa de aprendizagem}: 1e-2
    \begin{itemize}
        \item \textbf{Escalonamento}: redução de 10 vezes nas 25°, 35° e 45° épocas
        \item \textbf{Decaimento}: limitado a 1e-5
    \end{itemize}
    \item \textbf{Otimizador}: SGD (\textit{Stochastic Gradient Descent})
    \begin{itemize}
        \item \textit{Momentum}: 0,9
        \item \textit{Weight Decay}: 1e-5
    \end{itemize}
\end{itemize}

\begin{block}{Observações}
    Os detalhes de implementação acima foram utilizados em ambos os modelos e consideraram as escolhas de \textcite{Andre2021}.
\end{block}
       
\end{frame}

\begin{frame}
\framesubtitle{Função de Custo - Entropia Cruzada}

Mede a diferença entre a distribuição de probabilidade prevista (\(p_i\)) e o \textit{ground truth} dos rótulos para a classe (\(y_i\)), comumente expressa pela Equação \ref{eq:celoss} \cite{Celoss2019}.

% \begin{block}{Supondo que}
%     \(D = \{(x_1, y_1), ..., (x_i, y_i), ..., (x_M, y_M)\}\), onde \(y_i\) é um vetor representando o rótulo da \(i\)-ésima amostra \(x_i\), \(p_{ij}\) representa a probabilidade de que a amostra \(x_i\) seja atribuída à \(j\)-ésima classe, onde \(j\) varia de 1 a \(M\), sendo \(M\) o número total de classes.
% \end{block}

\begin{equation}\label{eq:celoss}
L_{CE} = -\frac{1}{M}\sum_{i=1}^{M}\left ( y_i^{T} log\left ( p_i \right ) \right )
\end{equation}

\noindent
onde \(M\) é o número total de classes, \(L_{CE}\) é a função de entropia cruzada, \(y_i^{T}\) é o vetor (\textit{ground truth}) transposto de \(y_i\), \(p\) é o vetor de probabilidades atribuídas a cada classe.

\end{frame}

\begin{frame}
\framesubtitle{Função de Custo - \textit{Focal Loss}}

É uma modificação da função de entropia cruzada que visa resolver o problema de desbalanceamento de classes, dando maior peso às classes minoritárias. A função é definida pela Equação \ref{eq:focal} \cite{FocalLoss2020}.

\begin{equation}\label{eq:focal}
FL(p_t) = -\sum_{i=1}^{C} (1 - p_{ti})^\gamma \cdot \log(p_{ti})
\end{equation}

\noindent
onde:
\begin{itemize}
    \item \(C\) é o número de classes.
    \item \(p_ti\) é a probabilidade prevista da classe verdadeira.
    \item \(\gamma\) é um parâmetro de foco ajustável.
    \item O termo \((1 - p_ti)^\gamma\) reduz o peso da perda para exemplos bem classificados, focando mais nos exemplos difíceis e mal classificados.
    \item Quando \(p_t\) está próximo de 1 (indicando uma previsão confiante e correta), a perda é reduzida.
\end{itemize}

Isso ajuda a reduzir o impacto de pixels fáceis de classificar e permite que o modelo se concentre mais em regiões desafiadoras.

\end{frame}

\section{Resultados e Discussão}\label{sec:resuldisc}



\subsection{Mais Exemplos de Figuras}\label{ssec:resuldisc1}

% \begin{frame}[allowframebreaks]
% As Figs.~\ref{fig:graficoxy1} e~\ref{fig:graficoxy2} são mais exemplos de figuras inseridas usando o ambiente \LaTeX\ \enquote{figure} e dispostas em duas colunas.
% \begin{columns}[T]
% \column{0.5\textwidth}
% \begin{figure}[!htb]
% \centering%
% \caption{Exemplo de legenda de figura.}%
% \label{fig:graficoxy1}
% \mode<presentation>{\includegraphics[width = \columnwidth]{./Figuras/graficoxy}}%% Modo apresentação: tamanho da figura
% \mode<article>{\includegraphics[width = 0.5\columnwidth]{./Figuras/graficoxy}}%% Modo artigo: tamanho da figura
% \source{Autoria própria.}
% \end{figure}
% \column{0.5\textwidth}
% \begin{figure}[!htb]
% \centering%
% \caption{Exemplo de legenda de figura.}%
% \label{fig:graficoxy2}
% \mode<presentation>{\includegraphics[width = \columnwidth]{./Figuras/graficoxy}}%% Modo apresentação: tamanho da figura
% \mode<article>{\includegraphics[width = 0.5\columnwidth]{./Figuras/graficoxy}}%% Modo artigo: tamanho da figura
% \source{autoria própria.}
% \end{figure}
% \end{columns}
% \framebreak%
% \mode<article>{\par}%% Modo artigo: quebra de parágrafo
% A Fig.~\ref{fig:mapacampus} apresenta um mapa com a localização dos câmpus da UTFPR\@.
% \begin{figure}[!htb]
% \centering%
% \caption{Mapa com a localização dos câmpus da UTFPR.}%
% \label{fig:mapacampus}
% \mode<presentation>{\includegraphics[width = 0.4\columnwidth]{./Figuras/mapacampus}}%% Modo apresentação: tamanho da figura
% \mode<article>{\includegraphics[width = 0.2\columnwidth]{./Figuras/mapacampus}}%% Modo apresentação: tamanho da figura
% \source{\textcite{UTFPR2018}.}
% \end{figure}
% \end{frame}

\section{Conclusões}\label{sec:concl}

\subsection{Descrição das Conclusões Obtidas}\label{ssec:concl1}

\begin{frame}
\begin{block}{Lista de conclusões}
\begin{itemize}
\item Conclusão 1.
\item Conclusão 2.
\item Conclusão 3.
\item Conclusão 4.
\item Conclusão 5.
\end{itemize}
\end{block}
\end{frame}

\mode<presentation>{%% Modo apresentação: referências
  \section{\refname}\label{sec:refs}%
  \frame[allowframebreaks]{%
    \framesubtitle{~}%
    \printbibliography[heading = none]%
  }%
}

\mode<article>{\printbibliography}%% Modo artigo: referências

\section{Agradecimentos}\label{sec:agrad}

\respnotice[Declaração de Responsabilidade]{O(s) autor(es) é(são) o(s) único(s) responsável(eis) pelas informações contidas neste documento.}

\begin{frame}{}{\mode<presentation>{~}}
Às organizações de fomento, pelo apoio recebido para o desenvolvimento deste trabalho e a participação neste evento:
\begin{center}
\includegraphics[height = 10mm]{./Logos/apoio-capes}
\hfill%
\includegraphics[height = 10mm]{./Logos/apoio-cnpq}
\hfill%
\includegraphics[height = 10mm]{./Logos/apoio-fa-gov-pr}
\hfill%
\includegraphics[height = 10mm]{./Logos/utfpr}
\end{center}
\mode<presentation>{%% Modo apresentação: agradecimentos e declaração de responsabilidade
  Aos presentes, pela atenção\sfootnote[frame]{\faStickyNoteO~\textbf{\respnoticetitle:}\space\MakeLowercase{\respnoticetext}}.%
}
\end{frame}

\mode<article>{%% Modo artigo: declaração de responsabilidade
  \section{\respnoticetitle}\label{sec:declar}%
  \respnoticetext%
}

%% Fim do documento
\end{document}


% \begin{frame}
% \begin{columns}[T]
% \column{0.5\textwidth}
% A Tab.~\ref{tab:Ldimens} é um exemplo de tabela inserida usando o ambiente \LaTeX\ \enquote{table} e numerada automaticamente.
% \begin{table}[!htb]
% \centering%
% \mode<presentation>{\scriptsize}%% Modo apresentação: tamanho de fonte
% \mode<article>{\small}%% Modo artigo: tamanho de fonte
% \caption{Exemplo de legenda de tabela.}%
% \label{tab:Ldimens}
% \begin{tabular*}{\columnwidth}{@{\extracolsep{\fill}}llll}
% \toprule
% $L$   & $L^2$     & $L^3$     & $L^4$     \\
% {[m]} & {[m$^2$]} & {[m$^3$]} & {[m$^4$]} \\
% \midrule
% 1     & 1         & 1         & 1         \\
% 2     & 4         & 8         & 16        \\
% 3     & 9         & 27        & 81        \\
% 4     & 16        & 64        & 256       \\
% 5     & 25        & 125       & 625       \\
% \bottomrule
% \addlinespace
% \end{tabular*}
% \source{autoria própria.}
% \end{table}
% \mode<article>{\par}%% Modo artigo: quebra de parágrafo
% Para gerar ou editar tabelas em \LaTeX, pode-se utilizar a ferramenta \enquote{\href{http://www.tablesgenerator.com/}{Tables Generator~\linkicon}}, entre outras.
% \column{0.5\textwidth}
% \begin{figure}[!htb]
% \centering%
% \caption{APA-Petrópolis.}%
% \label{fig:apa}
% \mode<presentation>{\includegraphics[width = 0.8\columnwidth]{./Figuras/apa}}%% Modo apresentação: tamanho da figura
% \mode<article>{\includegraphics[width = 0.4\columnwidth]{./Figuras/apa}}%% Modo artigo: tamanho da figura
% \source{\textcite{Google Earth, 2023}.}
% \end{figure}
% \end{columns}
% \end{frame}